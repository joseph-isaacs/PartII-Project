\documentclass[12pt,a4paper,twoside]{article}
\usepackage[pdfborder={0 0 0}]{hyperref}
\usepackage[margin=25mm]{geometry}
\usepackage{graphicx}
\usepackage{lmodern}
\usepackage{textcomp}
\usepackage[T1]{fontenc}
\usepackage{parskip}
\usepackage{verbatim}
\usepackage{amsfonts}
\usepackage[backend=biber,style=numeric,sorting=none]{biblatex}

\usepackage{xcolor}
\usepackage[normalem]{ulem}
\def\comment#1\done{{\color{blue}#1}}
\def\add#1\done{{\color{red}#1}}
\def\remove#1\done{{\sout{#1}}}
\def\change#1\to#2\done{{\sout{#1}\color{red}#2}}

\bibliography{ProRefs.bib}

\begin{document}


\begin{center}
  \Large
  Computer Science Tripos -- Part II -- Progress Report\\[4mm]
  \LARGE
  Haskell 98 to JVM bytecode compiler\\[4mm]

  \large
  J.~Isaacs, Sidney Sussex

  \texttt{josi2@cam.ac.uk}

  \today
\end{center}

\vspace{5mm}

\textbf{Project Supervisor:} R.~Kovacsics

\textbf{Director of Studies:} Dr J.~Fawcett 

\textbf{Project Overseers:} S.~Holden and S.~Teufel


My project is on schedule. I also have completed all work packages scheduled to be completed by 15/02/17.
I have also completed all work packages not requiring analysis or dissertation write up.



 \section*{Criteria Accomplished}

  The following criteria have been achieved:

  \begin{itemize}

    \item
      I have created unit tests which test simple Haskell programs, making sure that valid programs compile
      and produce the correct result and that invalid programs will not compile and produce errors.
      I have also checked that if the program is invalid the errors produced are related to the problems with
      the input program, where I use valid to mean syntatically correct and typeable.
      
      The unit tests I created help validate the individual stages in the compiler and test these
      stages in isolation. 
      I developed tests checking:
      \begin{itemize}
        \item Lexing and parsing of Haskell source.
        \item Checking the correctness of desugaring from a parse tree to an intermediate representation.
        \item If typeable and untypeable programs have correct typing judgments when run through 
          the type checker. Checking that for typeable programs the type checker returns the correct
          and most general type for that expression.
        \item The System F like core calculus, called Core, 
          outputted from the type-checking algorithm is annotated with the correct types.
        \item That the JVM bytecode outputted from the code generation stage of the compiler  
          will:
         \begin{itemize} 
            \item Run without unexpected runtime errors.
            \item Print correct values to stdout when the I/O monad is evaluated 
            \item Return the correct value if the program halts.
        \end{itemize}
      \end{itemize}

    \item As well fulfilling the core requirements I have implemented all of my extension tasks and have tests checking 
          that:
      \begin{itemize}
        \item Function inlining does not change the semantics of the program.
        \item Functions implemented in Java without side effect can be invoked by 
              Haskell code.
        \item Invoking monadic I/O functions such as \texttt{putInt} have the correct effects and 
              side effects.
      \end{itemize}

    \item I have implemented an analysis framework that will measure the run time of a given program
          multiple times, the data produced can then be further analysed.

  \end{itemize}


\end{document}
